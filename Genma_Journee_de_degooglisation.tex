\documentclass{beamer}
\mode<presentation> {
%\usetheme{Madrid}
%\usetheme{default}
\usepackage{color}
\definecolor{bottomcolour}{rgb}{0.21,0.11,0.21}
\definecolor{middlecolour}{rgb}{0.21,0.11,0.21}
\setbeamercolor{structure}{fg=white}
\setbeamertemplate{frametitle}[default]%[center]
\setbeamercolor{normal text}{bg=black, fg=white}
\setbeamertemplate{background canvas}[vertical shading]
[bottom=bottomcolour, middle=middlecolour, top=black]
\setbeamertemplate{items}[circle]
\setbeamertemplate{navigation symbols}{} %no nav symbols
\setbeamercolor{block title}{use=structure,fg=white,bg=structure.fg!50!red!50!blue!100!green}
\setbeamercolor{block body}{parent=normal text,use=block title,bg=block title.bg!5!white!10!bg,fg=white}
\setbeamertemplate{navigation symbols}{}
}

\usepackage{graphicx} 
\usepackage{booktabs} 
\usepackage[utf8]{inputenc}  
\usepackage[T1]{fontenc}  
\usepackage{geometry}     
\usepackage[francais]{babel} 
\usepackage{eurosym}
\usepackage{verbatim}
\usepackage{ragged2e}
\justifying

\input{cc_beamer}

\title[]{Les enjeux de la centralisation \\ des données personnelles \\ Sensibilisons aux risques,  \\proposons des alternatives.} 
\author{Ivaylo Ganchev, Genma}

\begin{document}

%% Titlepage
\begin{frame}
	\titlepage
	\vfill
	\begin{center}
		\CcGroupByNcSa{0.83}{0.95ex}\\[2.5ex]
		{\tiny\CcNote{\CcLongnameByNcSa}}
		\vspace*{-2.5ex}
	\end{center}
\end{frame}

%----------------------------------------------------------------------------------------

\begin{frame}
\frametitle{\includegraphics[scale=0.4]{./images/Genma.jpg} \ \ \  A propos de moi  }
\begin{columns}[c] 

\column{.55\textwidth} 
\textbf{Où me trouver sur Internet?}
\begin{itemize}
\item Le Blog de Genma : http://genma.free.fr
\item Twitter : http://twitter.com/genma
\end{itemize}

\textbf{Ce que je fais?}
\\ Plein de choses dont:
\begin{itemize}
\item Organisateur de cafés vie privée,
\item Bénévole Framasoft. Mozilla...
\end{itemize}

\column{.5\textwidth} 
\includegraphics[width=5cm,height=5cm]{./images/blog.png} 
\end{columns}
\end{frame}


%----------------------------------------------------------------------------------------
\begin{frame}
\Huge{\centerline{Nous visons dans}}
\Huge{\centerline{un monde de plus en plus centralisé}}
\end{frame}


%------------------------------------------------
\begin{frame}
\frametitle{Enjeux}

\begin{block}{Les enjeux}
\justifying{
\begin{itemize}
\item Concentration des acteurs d’Internet autour de sillos
\item Une centralisation nuisible (frein à l'innovation)
\item  Les utilisateurs de ces services derniers ne contrôlent plus leur vie numérique
\end{itemize}
\begin{center}
\includegraphics[scale=0.3]{./images/fight.png}
\end{center}
}
\end{block}
Pius de détail sur \url{http://degooglisons-internet.org/}
\end{frame}

%------------------------------------------------
\begin{frame}
\frametitle{Les GAFAM}

\begin{block}{Les dangers}
\justifying{
Les services en ligne toujours plus centralisés de géants tentaculaires comme Google, Amazon, Facebook, Apple ou Microsoft (GAFAM) mettent en danger nos vies numériques.
\begin{itemize}
\item Espionnage
\item Vie privée
\item Centralisation
 \item Fermeture
\end{itemize}
\begin{center}
\includegraphics[scale=0.3]{./images/steve.png}\quad
\includegraphics[scale=0.3]{./images/legion.png}
\end{center}
}
\end{block}
Pius de détail sur \url{http://degooglisons-internet.org/}
\end{frame}

%----------------------------------------------------------------------------------------
\begin{frame}
\Huge{\centerline{Vous ne me croyez pas?}}
\Huge{\centerline{Vous n'avez rien à cacher?}}
\end{frame}

%------------------------------------------------
\begin{frame}
\frametitle{L'affirmation Je n'ai rien à cacher}
\Huge{\centerline{Donnez moi vos mots de passe...}}
\end{frame}


%----------------------------------------------------------------------------------------
\begin{frame}
\frametitle{L'identité numérique c'est quoi?}
\begin{block}{Définition}
\begin{itemize}
\justifying{
\item L'identité numérique, c'est l'ensemble des données publiques que l'on peut trouver sur Internet et rattacher à une personne.
\item C'est la fameuse e-reputation.
}
\end{itemize}
\end{block}
\end{frame}

\begin{frame}
\frametitle{L'image que je donne de moi}
\justifying{
\begin{block}{Googler "son nom".}
\begin{itemize}
\item Les résultats apparaissant sont-ils bien ce que l'on souhaite?
\end{itemize}
\end{block}
}
\begin{center}
\includegraphics[scale=0.3] {./images/Google01.png}
\\
\includegraphics[scale=0.3] {./images/Google02.png}
\end{center}
\end{frame}

\begin{frame}
\frametitle{Adage}
\begin{block}{Les paroles s'envolent, les écrits restent}
\begin{itemize}
\justifying{
\item Cet adage est encore plus vrai avec Internet.
\item  Il faut partir du principe que ce que l'on dit sera toujours accessible, même des années après.
\item Tout ce qui est sur Internet est public ou le sera (même si c'est "privé". Les conditions d'utilisation évoulent. Cf.Facebook).
\item Il ne faut donc pas abuser de la liberté d'expression et rester respectueux des lois en vigueurs.
}
\end{itemize}
\end{block}
Et c'est encore plus facile de retrouver toutes ces infomations si elles sont en un seule et même endroit...
\end{frame}


%----------------------------------------------------------------------------------------
\begin{frame}
\begin{center}
\Huge{Toutes ces informations que l'on donne... }
\Huge{Volontairement...}
\Huge{ou pas}
\end{center}
\end{frame}

%------------------------------------------------
\begin{frame}
\frametitle{Rien qu'avec les services Google}

\begin{itemize}
\justifying{
\item Google Search
\item GMail 
\item Google Analytics 
\item Google Maps 
\item Smartphone Android 
\item Google Calendar 
\item Google Wallet 
\item Google Docs et Drive 
\item Google Chrome, navigateur 
\item Google Photos 
\item Youtube 
\item ...
}
\end{itemize}

\end{frame}


%------------------------------------------------
\begin{frame}
\frametitle{Les applications pour Smartphone}
\begin{block}{75 \% des applis mobiles collectent des données personnelles}
\justifying{
Selon une étude réalisée par la CNIL et ses homologues européennes, avec l'examen de plus de 1200 applications mobiles,
 les trois quarts des applis installées sur les smartphones collectent des données personnelles, et la plupart le font avec une information insuffissante du consommateur.
}
Source \url{http://www.numerama.com}
\end{block}
Est-il normal qu'une application lampe de poche est besoin d'accéder au carnet d'adresses?
\end{frame}

\begin{frame}
\frametitle{Illustrations}
\begin{center}
\includegraphics[scale=0.3]{./images/AXA_assurance_gps.jpg}
\end{center}
\end{frame}

%------------------------------------------------
\begin{frame}
\frametitle{Une carte de géolocalisation des photos de chat 1/2}
\justifying{
Un artiste s'est amusé à récupérer toutes les photos de chats qu'il trouvait sur leweb et en se basant sur les données de géolocalisation de ces photos, à placer celles-ci sur une carte du monde. Voici ce que ça a donné :
}
\begin{center}
\includegraphics[scale=0.3]{./images/Chat_geolocalisaion.png}
\end{center}
\end{frame}
%------------------------------------------------
\begin{frame}
\frametitle{Une carte de géolocalisation des photos de chat 2/2}
On sait donc quel chat habite où - qui a un chat.
\\~\\
Réfléchissez-y. \textbf{Et si quelqu'un fasait la même chose mais avec d'autres photos...}
\end{frame}

%------------------------------------------------
\begin{frame}
\frametitle{Les objets liés à la santé}

\begin{block}{ les « trackers d’activité physique » }
\begin{itemize}
\justifying{
\item  ils comptent les calories dépensées pendant la journée en fonction de nos mouvements
\item les données sont envoyés à un serveur centralisé
}
\end{itemize}
\end{block}

\begin{block}{Apple, avec HealthKit}
\begin{itemize}
\justifying{
\item Apple veut les vendre aux assurances pour permettre aux assureurs de conditionner des remboursements ou divers avantages tarifaires à un comportement sanitaire exemplaire, surveillé par la collecte de données en temps réel.
}
\end{itemize}
\end{block}
\end{frame}
%------------------------------------------------
\begin{frame}
\frametitle{Illustrations}
\begin{center}
\includegraphics[scale=0.5]{./images/Apple_IOS_DonneesMedicales.png}
\end{center}
\end{frame}

%------------------------------------------------
\begin{frame}
\frametitle{Les assurances et les objets connectés}
\justifying{
Les applications pour smartphone tout comme les objets connectés (bracelets)captent votre rythme cardiaque, analysent les photos de vos repas ou comptent le nombre de pas que vous effectuez dans une journée.
\\~\\
Le but de l'accumulation de toutes ces données afin aux compagnies d’assurances, publicitaires et autres professionnels de la « santé ».
\\~\\
Le risque est de payer plus cher son assurance car on n'a pas ce type d'objet. Donc on acceptera d'être surveillé pour des raisons de baisse de prix....
}
\end{frame}

%----------------------------------------------------------------------------------------
\begin{frame}
\begin{center}
\Huge{Conséquences de l'usage de toutes ces données }
\end{center}
\end{frame}

%----------------------------------------------------------------------------------------
\begin{frame}
\begin{center}
\Huge{A une échelle BigData}
\end{center}
\end{frame}

%------------------------------------------------
\begin{frame}
\frametitle{GoogleFlu et Google trends}
\begin{block}{Présentation de Google Flux sur leur site}
\begin{itemize}
\justifying{
\item Nous avons remarqué que certains termes de recherche étaient des indicateurs efficaces de la propagation de la grippe. Google Flu rassemble donc des données de recherche Google pour fournir une estimation quasiment en temps réel de cette propagation à l'échelle mondiale. \url{http://www.google.org/flutrends/}
\item On peut aussi voir les recherches temps réels, par mot clef, pays... \url{https://www.google.com/trends/}
}
\end{itemize}
\end{block}
\justifying{
Tout cela ne me laisse pas indifférent quand à \textbf{ la puissance de Google}. Et vous?
}
\end{frame}

%----------------------------------------------------------------------------------------
\begin{frame}
\Huge{\centerline{A l'échelle de l'individu...}}
\end{frame}


%----------------------------------------------------------------------------------------
\begin{frame}
\begin{center}
\Huge{Sur Internet, si c'est gratuit, c'est vous le produit }
\end{center}
\end{frame}
%----------------------------------------------------------------------------------------
\begin{frame}
\frametitle{Qu'est-ce que le pistage ?}
\begin{block}{Le pistage sur Internet}
\begin{itemize}
\justifying{
\item Le pistage est un terme qui comprend des méthodes aussi nombreuses et variées que les sites web, les annonceurs et d'autres utilisent pour connaître vos habitudes de navigation sur le Web. 
\item  Cela comprend des informations sur les sites que vous visitez, les choses que vous aimez, n'aimez pas et achetez. 
\item Ils utilisent souvent ces données pour afficher des pubs, des produits ou services spécialement ciblés pour vous. 
}
\end{itemize}
\end{block}
\end{frame}

%----------------------------------------------------------------------------------------
\begin{frame}
\frametitle{Comment est-on tracké?}

\justifying{
\begin{block}{Toutes les publicités nous espionnent}
\begin{itemize}
\item Le bouton Like de Facebook : il permet à FaceBook de savoir que vous avez visité ce site, même si vous n'avez pas cliqué sur ce bouton.
\item Même si vous vous êtes correctement déconnecté de Facebook.
\item De même pour le bouton le +1 de Google, les scripts de Google Analytics, 
\item Tous les publicité, Amazon...
\end{itemize}
\end{block}
}
\begin{center}
\includegraphics[scale=0.3] {./images/Facebook_like.png}
\end{center}
\end{frame}

%----------------------------------------------------------------------------------------
\begin{frame}
\Huge{\centerline{Existe-t-il une solution?}}
\end{frame}


%----------------------------------------------------------------------------------------
\begin{frame}
\Huge{\centerline{Framasoft}}
\Huge{\centerline{(une solution parmi d'autres)}}
\begin{center}
\includegraphics[scale=0.6]{./images/pingouinVolantRefait.jpg}
\end{center}
\end{frame}

%------------------------------------------------
\begin{frame}
\frametitle{Framasoft c'est quoi?}

\begin{block}{Présentation}
\justifying{
Framasoft est une association francophone de promotion et diffusion des logiciels libres. C'est aussi
\begin{itemize}
\item Un réseau dédié à la promotion du « libre » en général et du logiciel libre en particulier.
\item De nombreux services et projets innovants mis librement à disposition du grand public.
\item Une communauté de bénévoles soutenue par une association d’intérêt général.
\end{itemize}
}
\end{block}
Plus de détail sur \url{http://www.framasoft.org}
\end{frame}


%----------------------------------------------------------------------------------------
\begin{frame}
\Huge{\centerline{Framasoft et la dégooglisation}}
\begin{center}
\includegraphics[scale=0.6]{./images/cloud.jpg}
\end{center}
\end{frame}

\begin{frame}
\begin{center}
\includegraphics[scale=0.5]{./images/DegooglisonsInternet.jpg}
\end{center}
\end{frame}
%------------------------------------------------
\begin{frame}
\frametitle{Dégooglisons Internet}

\begin{block}{Framasoft lance une campagne d’envergure : Dégooglisons Internet. }
\justifying{
Mise en place de plus de services L.E.D.S. (Libres, Éthiques, Décentralisés et Solidaires). 
}
\end{block}
\end{frame}

%------------------------------------------------
\begin{frame}
\frametitle{Dégooglisons Internet}

\begin{block}{L'objectif}
\justifying{
Les logiciels libres, de par leur nature ouverte, sont les seuls à vraiment garantir le respect de votre vie privée.
\\~\\
Il faut donc installer, face à chaque service propriétaire, un service hébergé par les soins de Framasoft. 
\\~\\L’association s’inscrit dans un contexte d’ouverture en encourageant les projets qui participeront à cet effort d’émancipation des « grands de l’Internet ».
}
\end{block}
Plus de détail sur \url{http://degooglisons-internet.org/}
\end{frame}


%------------------------------------------------
\begin{frame}
\frametitle{Les propositions de Framasoft}

\begin{block}{Ce que Framasoft propose}
\justifying{
Framasoft souhaite faire face à ces dangers menaçant nos vies numériques en proposant des services
\begin{itemize}
\item  libres,
\item éthiques, 
\item décentralisés 
\item et solidaires.
\end{itemize}
\begin{center}
\includegraphics[scale=0.3]{./images/potion.png}
\end{center}
}
\end{block}
Plus de détail sur \url{http://degooglisons-internet.org/}
\end{frame}

%------------------------------------------------
\begin{frame}
\frametitle{Concrètement, ce que fait Framasoft}

\begin{block}{Concrètement}
Le projet « Dégooglisons Internet » - qui ne concerne d'ailleurs pas que Google - consiste à proposer des services alternatifs face à un maximum de services que nous évaluons comme menaçants pour nos vies numériques.
\justifying{
\begin{itemize}
\item Des services sont libres, gratuits, ouverts à tous (dans la limite de nos capacités techniques et financières),
\item Promotion de l'auto-hébergement, 
\item Proposer une alternative.
\end{itemize}
}
\end{block}
Plus de détail sur \url{http://degooglisons-internet.org/}
\end{frame}

%----------------------------------------------------------------------------------------
\begin{frame}
\Huge{\centerline{Conclusion...}}
\Huge{\centerline{mais c'est loin d'être fini}}
\end{frame}

%------------------------------------------------
\begin{frame}
\frametitle{Les cafés vie privée}

\begin{center}
\includegraphics[scale=0.4]{./images/LogoCafeViePrivee.jpg}
\end{center}
\end{frame}

%----------------------------------------------------------------------------------------
\begin{frame}
\Huge{\centerline{Merci de votre attention}}
\Huge{\centerline{Place aux questions}}
\end{frame}

\end{document}
